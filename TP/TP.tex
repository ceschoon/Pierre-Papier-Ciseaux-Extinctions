\documentclass[openany,a4paper,12pt]{article}
%%%%%%%%%%%%%%%%%%%%%%%%%%%%%%%  * PACKAGES *  %%%%%%%%%%%%%%%%%%%%%%%%%%%%%%%%%%%%

%% DOC   --------------------------------------------------------------------------

\usepackage[utf8]{inputenc}
\usepackage[francais]{babel}
\usepackage[top=3cm, bottom=3cm, left=3cm, right=3cm]{geometry}
%\usepackage{charter}
%\usepackage{fancyhdr}
%\usepackage{lastpage}

\setlength{\parindent}{5mm}
\setlength{\parskip}{5mm plus 1mm minus 1mm}
\usepackage{indentfirst}

%% FONTS ET SPACES   --------------------------------------------------------------

%\usepackage{setspace}
%\onehalfspacing

%\usepackage{times}
%\usepackage{fontspec}
%\setmainfont{Times}
%\fontsize{12pt}
%\fontfamily{ptm}

%\usepackage{libertine}
%\usepackage{biolinum}
%\usepackage{libertinust1math}
%\usepackage[T1]{fontenc}

%% OTHER PACKAGES   ---------------------------------------------------------------

\usepackage{physics}
\usepackage{chemformula}
%\usepackage{algorithmicx}
%\usepackage{pythonhighlight}

\usepackage{amsmath}
\usepackage{amssymb}
%\usepackage{amsthm}
%\numberwithin{equation}{section}
%\usepackage{siunitx}
%\usepackage[squaren,Gray]{SIunits}

\expandafter\def\expandafter\normalsize\expandafter{%
	\normalsize
	\setlength\abovedisplayskip{5mm}
	\setlength\belowdisplayskip{5mm}
	\setlength\abovedisplayshortskip{5mm}
	\setlength\belowdisplayshortskip{5mm}
}


\usepackage{cancel}

\usepackage{wrapfig}
\usepackage{graphicx}
%\usepackage{booktabs}
%\usepackage{framed}

%\usepackage{environ}
%\usepackage{tikz}
%\usetikzlibrary{calc}

\usepackage{caption}
\usepackage{subcaption} % two figures side by side
%\usepackage{placeins}
\usepackage[section]{placeins} % barrier to not have a figure in the wrong section
%\usepackage{longtable} % needed for long tables over pages
%\usepackage{enumerate} % needed for some options in enumerate

\usepackage{todonotes} % needed for todos
%\usepackage{makeidx} % needed for creating an index
%\makeindex


%\usepackage{listings}

%% LINKS   ------------------------------------------------------------------------

%\usepackage[hidelinks=true,colorlinks=true,breaklinks]{hyperref}
\usepackage{hyperref} % boxes
%\usepackage{xcolor}
%\definecolor{c1}{rgb}{0,0,1} % blue
%\definecolor{c2}{rgb}{0,0.25,0.75} % green blue
%\definecolor{c3}{rgb}{0.25,0,0.75} % red blue
%\hypersetup{
%	linkcolor={c1}, % internal links
%	citecolor={c1}, % citations
%	urlcolor={c3}, % external links/urls
%	runcolor={c1} % executable links
%}
%\usepackage[hyphenbreaks]{breakurl}


%% TITLE
\title{
\textsc{Travail personnel}\\ 
Processus Stochastiques en Physique\\
PHYS-F446\\
\rule{\linewidth}{1pt} \\
\vspace{3mm}
Extinctions dans un modèle de prédation cyclique \\
de type "pierre papier ciseaux"\\
\rule{\linewidth}{1pt}
}
\author{Cédric \textsc{Schoonen}}

%% BIBLIOGRAPHY
\input{settings/bibliography}
\bibliography{literature/library}

%% OTHER


%%%%%%%%%%%%%%%%%%%%%%%%%%%%%%%%%%%%%%%%%%%%%%%%%%%%%%%%%%%%%%%%%%%%%


\begin{document}

\maketitle
\vspace{1cm}

\tableofcontents



%%%%%%%%%%% Plan %%%%%%%%%%%

% Présentation du système 
%   "eq chimiques" 
% Dynamique macroscopique (déterministe)
%   equations dynamique macroscopique
%   portrait de phase
%   invariants
%   bihamiltonien ??
% Dynamique microscopique (stochastique)
%   eq maitresse
%   dev de Kramers-Moyal et eq Fokker-Planck (!echelle de temps!)
%   SDE
%   croissance moyenne du second invariant ??
% Réduction à un processus radial
%   changmt var pour rho
%   stochastic averaging
%   SDE 1D
%   calcul numérique de D(rho)
% Problème d'échappement région 1D
%   dév section 5.2 Gardiner
% Temps moyens d'extinction (processus radial)
%   application formules d'échappement au temps d'extinction
%   intégrations numériques
%   résultats, comparaison interp-approx-simul
% Simulation du système
%   méthode avec algo gillespie
%   trajectoire dans portrait de phase
%   résultats pour plusieurs espèces ??

%%%%%%%%%%%%%%%%%%%%%%%%%%%%

	
	
%\begin{figure}
%	\centering
%	\begin{subfigure}{.33\textwidth}
%		\centering
%		\includegraphics[width=\linewidth]{figures/p2_avant_bp}
%		\caption{}
%	\end{subfigure}%
%	\begin{subfigure}{.33\textwidth}
%		\centering
%		\includegraphics[width=\linewidth]{figures/p2_1bp1}
%		\caption{}
%	\end{subfigure}%
%	\begin{subfigure}{.33\textwidth}
%		\centering
%		\includegraphics[width=\linewidth]{figures/p2_1bp2}
%		\caption{}
%	\end{subfigure}
%	\caption{Représentation graphique des actions des deux applications $b_p$ obtenues à la question 2.3. La figure (a) représente les points du carré unité avant l'action des $b_p$, la figure (b) les même points après l'action de l'application préservant les aires et la figure (c) indique l'action de l'application ne préservant pas les aires.}
%	\label{fig:p2-graphesbp}
%\end{figure}

		
\printbibliography

\end{document} 

